
We have studied the transcriptional signatures of two forms of differentiation that mouse ES cells can undergo. The first is the spontaneous differentiation that occurs randomly in some cells within populations maintained in stem cell media. These spontaneously differentiated cells make up a usually stable fraction of the entire population. The second type of differentation we looked at involved removing the molecular factors necessary to maintain the ES cell state from the culture medium. After removal, the entire population slowly differentiates.

We used a VNP fluorescent reporter of Nanog expression to mark and sort cells that have deviated from pluripotency (VNP-) from fully pluripotent cellls (VNP+) in medium containing the 2i inhibitors and LIF (Figure 1a,b). We removed the 2i and LIF for a day from the culture to study culture-wide differentiation. We used RNA-Seq to quantify the gene expression changes associated to these two types of deviation from pluripotency, and the fold changes for a few known markers of pluripotency and early lineage specification are shown in Figure 1c. Pluripotency genes are unsuprisingly down in both the differentiated culture and in spontaneously deviated cells. On the other hand, the spontaneously differentiated cells show stronger up-regulation of lineage markers, particularly mesoderm, than culture-wide differentiation. Hox genes are also uniformly and strongly up-regulated in spontaneous differentiation but if anything down-regulated on culture-wide differentiation.

Looking genome wide, we identified ~3000 genes that we are confident (at a 10\% false discovery rate) are differentially expressed between pluripotent and spontaneously differentiated cells. Approximately the same number are differentially expressed when the culture undergoes a day of differentiation in the absence of 2i+LIF, and there is considerable overlap between the two sets. We analyzed the composition of these gene sets in terms of their molecular function (gene ontology). We used a ``greedy'' method to determine the set of molecular function categories that best describes the collection of differentially expressed genes and to assign each gene to a category. We found (Fig. 2c) that embryo development is the most important molecular function of the differentially expressed genes, closely followed by cell adhesion, a function that in earlier studies has been associated with early developmental changes. You can also notice in Fig. 2c that the set of genes up-regulated in spontaneously differentiated cells is more heavily weighted towards embryo development than the set of genes up-regulated upon culture-wide differentiation. 

Not only are embryo development and lineage marker genes highly overrepresented amongst up-regulated genes, they are also amongst the most strongly affected. As shown in Fig. S4, the lineage-associated genes from Fig. 1C are found towards the extremes of the foldchange-expression level distribution of our hits, and half of the genes with comparably extreme changes are also annotated for roles in development and pattern specification. The expression of lineage-specific markers is therefore one of the strongest features in the transcriptome signature of spontaneously differentiated cells.

Of the more than 1300 genes that we are confident are differentially expressed both upon spontaneous and culture-wide differentiation, about one third are differentially expressed in ``incoherent'' directions. These 459 genes are either up-regulated on spontaneous differentiation and down-regulated on culture-wide differentiation or vice versa. The former scenario includes many lineage markers, and the Hox genes, which are slightly repressed on culture-wide differentiation but are considerably more highly expressed in spontaneously differentiated cells than in pluripotent cells. 


We were intrigued by the lineage marker signature in spontaneously differentiated cells and we investigated how that is manifest at the single cell level to answer a few questions. First, we observed only mild downward fold-changes up to a factor of 2 for pluripotency genes in our Nanog:VNP(-) population, and we suspected that the fold-changes are artificially dampened by reporter false negative cells.  Regarding lineage-associated genes, although they constitute the  strongest expression signature of spontaneously differentiated cells, we observed upwards fold-changes in Nanog:VNP(-) mostly between factors of 4 and 8. Underlying this could be low-level but somewhat uniform expression of these genes in large fractions of Nanog:VNP(-) cells, at the level of few transcripts per cell. Instead, we considered it more likely that these lineage-associated gene may be expressed in very small and possibly exclusive subpopulations within Nanog:VNP(-) that represent relatively advanced stages of lineage commitment. We were also very interested to see if lineage-associated gene expression arises mostly in cells that have been thoroughly depleted of pluripotency gene expression.

... fish part from our earlier submission, nearly as-is, but much less stressing the reversibility question ...

# Discussion

It is interesting that within stem-ness promoting cultures, spontaneously differentiated cells arise that are characterized mainly by lineage marker and developmental gene expression. In the case of culture-wide differentiation, removal of the factors in the media that maintain the stem cell state leads to cells escaping the pluripotent state and understandably evolving along lineage paths. It is nontrivial that spontaneously differentiated cells, which still experience the media factors that keep cells as stem cells in the first place, arrive at states defined by lineage and developmental marker expression. 